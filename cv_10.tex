%%%%%%%%%%%%%%%%%%%%%%%%%%%%%%%%%%%%%%%%%
% Friggeri Resume/CV
% XeLaTeX Template
% Version 1.0 (5/5/13)
%
% This template has been downloaded from:
% http://www.LaTeXTemplates.com
%
% Original author:
% Adrien Friggeri (adrien@friggeri.net)
% https://github.com/afriggeri/CV
%
% License:
% CC BY-NC-SA 3.0 (http://creativecommons.org/licenses/by-nc-sa/3.0/)
%
% Important notes:
% This template needs to be compiled with XeLaTeX and the bibliography, if used,
% needs to be compiled with biber rather than bibtex.
%
%%%%%%%%%%%%%%%%%%%%%%%%%%%%%%%%%%%%%%%%%

\documentclass[]{friggeri-cv} % Add 'print' as an option into the square bracket to remove colors from this template for printing

\addbibresource{bibliography.bib} % Specify the bibliography file to include publications

\begin{document}

\header{Alvaro }{Segura Del Barco}{Networks \& systems administrator} % Your name and current job title/field

%----------------------------------------------------------------------------------------
%	SIDEBAR SECTION
%----------------------------------------------------------------------------------------

\begin{aside} % In the aside, each new line forces a line break
\section{contact}
San José
Costa Rica
~
+506 83185468
~
\href{mailto:alvarosb@gmail.com}{alvarosb@gmail.com}
\href{http://wp.me/4uJoK}{http://wp.me/4uJoK}
\href{http://git.io/DM8URA}{http://git.io/DM8URA}
\href{http://rngkll.github.io}{http://rngkll.github.io}
\section{languages}
spanish mother tongue
english fluency
\section{programming}
%{\color{red} $\varheartsuit$} 
Python,
Shell scripting, C,
C++, Tcl, \LaTeX, html,
Verilog, java
\end{aside}



%----------------------------------------------------------------------------------------
%	WORK EXPERIENCE SECTION
%----------------------------------------------------------------------------------------

\section{experience}

\begin{entrylist}

\entry
{2015--2018}
{Akamai technologies}
{Escazu, Costa Rica}
{\emph{Solutions architect} \\
Create, modify and consult different customers, first in the media industry the as a Carrier product integration specialist, working with Licensed CDN and DNS systems.\\
Manage the innovation laboratory in Costa Rica, represent Akamai in the "Espacios de innovación y tecnología aplicada" \href{https://www.cinde.org/}{CINDE} (Costa Rican investment promotion agency) program.

Detailed achievements:
\begin{itemize}
\item Developed a testing framework to test Akamai's customers configurations.
\item Developed tools to push media traffic to the platform to test customers configurations.
\item SA assigned to the AURA project, Licensed CDN Akamai product.
\item Part of the starting group of the X-Akamai-Project initiative.
\item Akamai EDGE 2017 speaker and expert bar consultant.
\item Akamai AIRs 2018 presenter.
\item Successfully integrated multiple LCDN and AnswerX with the carriers team.
\end{itemize}
}

%------------------------------------------------
\entry
{2014--2015}
{Softtek}
{Heredia, Costa Rica}
{\emph{Embedded developer} \\
C developer for the HP wired networking devices

Detailed achievements:
\begin{itemize}
\item Developed debugging commands for HP networking devices.
\item Migrated Uboot for the CPU architecture in the network devices.
\end{itemize}
}

%------------------------------------------------
\entry
{2013--2014}
{Softtek}
{Heredia, Costa Rica}
{\emph{Senior QA Engineer} \\
Triage and Test-case programming for the upcoming software for HP wired networking devices.\\
Rehab(rewrite test case according to programming standards).

Detailed achievements:
\begin{itemize}
\item Developed tools to reduce triage times
\begin{itemize}
\item Tool to link a list of issues to a specific bug ID
\item Tool to get information of the issues from CIT(bug tracking interface) and generate list of IDs to link
\end{itemize}
\item Developed tools to reduce rehab process times
\begin{itemize}
\item Tool to parse the test case steps and update the list in the beginning of the test case 
\end{itemize}
\end{itemize}
}


%------------------------------------------------
\entry
{2010--2013}
{Abax Asesores S.A.}
{San José, Costa Rica}
{\emph{Network and Systems administrator}\\
Platforms integration and support for clients, mostly based on free software Internal infrastructure support.
LAMP, LAPP, VPN, DNS and DHCP servers installation and configuration, Web stack optimization.

Detailed achievements:
\begin{itemize}
\item Real time monitoring platform for in-house and clients servers, using Nagios and Icinga
\begin{itemize}
\item Platform integration
\item Nagios Plug-in development using python and shell scripting
\end{itemize}
\item High traffic optimization for several mayor communication clients using tools as Varnish reverse proxy and memcache
\end{itemize}
}
\end{entrylist}
\begin{entrylist}
%------------------------------------------------
\entry
{2008-2011}
{Universidad de Costa Rica}
{San José, Costa Rica}
{\emph{Networks and systems administrator}\\
Network design and support, personal computers supporting GNU/Linux and Microsoft Windows, Gnu/Linux Servers support.

Detailed achievements:
\begin{itemize}
\item Network redesign
\begin{itemize}
\item Network segmentation using VLANs to separate users and traffic
\item Public network creation with private IP addresses for regular users and laboratories
\end{itemize}
\item Virtual server implementation
\begin{itemize}
\item Migrated all the physical servers to virtual servers first using Xen and then Kvm
\item Improve servers uptime by migrating virtual servers from their physical hosts
\item Created separated virtual machines for needed services as web cache, mail, proxy, file server, DHCP, DNS
\item Installation of a debian distribution mirror with support for most of the architectures at http://debian.emate.ucr.ac.cr to use in laboratory with powerpc machines
\end{itemize}
\end{itemize}
}
%------------------------------------------------
\end{entrylist}

\begin{entrylist}
\entry
{2008}
{Colegio de Biólogos de Costa Rica}
{San José, Costa Rica}
{\emph{Solutions integration}\\
Virtual Servers implementation using Xen Virtualization software, various Services implementation such as DNS , DHCP, MySQL and WEB, using GNU/Linux systems.
}

\entry
{2006-2008}
{Electric engineering, Universidad de Costa Rica}
{San José, Costa Rica}
{\emph{Network assistant}\\
Help to develop and manage network infrastructure.
}
%------------------------------------------------
\end{entrylist}
\pagebreak
%----------------------------------------------------------------------------------------
%	Freelance section
%----------------------------------------------------------------------------------------
\section{freelance experience}
\begin{entrylist}
%------------------------------------------------
\entry
{2005-2014}
{}
{Costa Rica}
{
Solutions in networking and servers for several clients as:
\begin{itemize}
\item Fundación Acceso
\begin{itemize}
\item Design and install the laboratories for the "Superadas" project that helps children in social risk
\end{itemize}
\item Fundación Demuca
\item Colegio La Salle
\item Colegio de Biólogos de Costa Rica
\item Universidad de Costa Rica
\end{itemize}
}
%------------------------------------------------
\end{entrylist}

%----------------------------------------------------------------------------------------
%	EDUCATION SECTION
%----------------------------------------------------------------------------------------

\section{education}
\begin{entrylist}
%------------------------------------------------
\entry
{2005}
{Cisco Certified Networking Associate (CCNA)}
{Network academy, UCR}
{Modules 1-4, in process of certification}
%------------------------------------------------
\entry
{2004}
{Electric Engineering}
{Universidad De Costa Rica}
{Senior student of the computers and networks study plan}
%------------------------------------------------
\entry
{2001}
{Electronic Technician}
{Instituto Tecnológico de Costa Rica}
{}
%------------------------------------------------
\end{entrylist}

%----------------------------------------------------------------------------------------
%	COMMUNICATION SKILLS SECTION
%----------------------------------------------------------------------------------------

%\section{communication skills}
%
%\begin{entrylist}
%%------------------------------------------------
%\entry
%{2011}
%{Oral Presentation}
%{California Business Conference}
%{Presented the research I conducted for my Masters of Commerce degree.}
%%------------------------------------------------
%\entry
%{2010}
%{Poster}
%{Annual Business Conference, Oregon}
%{As part of the course work for BUS320, I created a poster analyzing several local businesses and presented this at a conference.}
%%------------------------------------------------
%\end{entrylist}

%----------------------------------------------------------------------------------------
%	INTERESTS SECTION
%----------------------------------------------------------------------------------------

\section{interests}

\textbf{professional:} open hardware, electronics, free software, networks, cloud computing, virtulization, programming, \textbf{personal:} basketball, snorkeling , cooking, running, jaquerespeis.org.

%%----------------------------------------------------------------------------------------
%%	PUBLICATIONS SECTION
%%----------------------------------------------------------------------------------------
%
%\section{publications}
%
%\printbibsection{article}{article in peer-reviewed journal} % Print all articles from the bibliography
%
%\printbibsection{book}{books} % Print all books from the bibliography
%
%\begin{refsection} % This is a custom heading for those references marked as "inproceedings" but not containing "keyword=france"
%\nocite{*}
%\printbibliography[sorting=chronological, type=inproceedings, title={international peer-reviewed conferences/proceedings}, notkeyword={france}, heading=subbibliography]
%\end{refsection}
%
%\begin{refsection} % This is a custom heading for those references marked as "inproceedings" and containing "keyword=france"
%\nocite{*}
%\printbibliography[sorting=chronological, type=inproceedings, title={local peer-reviewed conferences/proceedings}, keyword={france}, heading=subbibliography]
%\end{refsection}
%
%\printbibsection{misc}{other publications} % Print all miscellaneous entries from the bibliography
%
%\printbibsection{report}{research reports} % Print all research reports from the bibliography
%
%%----------------------------------------------------------------------------------------

\end{document}
